This final chapter moves from the presentation of results to their interpretation and synthesis, drawing together the key insights from the research. It begins by summarizing the principal findings of the comparative study, directly addressing the research questions posed in the introduction. The discussion then explores the broader implications of these findings for financial practitioners and academic researchers, identifying scenarios where each calibration method may be preferred. The chapter also acknowledges the inherent limitations of this study and, based on the outcomes and unanswered questions, proposes promising directions for future research in the field of model calibration and machine learning in finance.
\begin{itemize}
    \item \textbf{Restate Thesis Aim:} Briefly reiterate the primary objective of the research.
    \item \textbf{Summarize Key Findings:} Condense the most significant results from the "Results" section regarding the comparative performance (speed, accuracy, robustness) of traditional versus ML-based calibration.
    \item \textbf{Discuss Implications:} Elaborate on what the findings mean for financial practitioners and academic researchers.
    \begin{itemize}
        \item Identify scenarios or contexts where machine learning approaches might offer distinct advantages over traditional methods, and vice-versa.
        \item Discuss whether specific market conditions favor one method over the other.
        \item Highlight the practical benefits, such as improved risk management, faster pricing of complex derivatives, or more robust model fitting.
    \end{itemize}
    \item \textbf{Limitations:} Acknowledge any constraints or limitations of the study (e.g., specific dataset size, focus on a one-factor model, inherent simplifications, computational resources).
    \item \textbf{Future Work:} Propose potential directions for further research. This could include exploring multi-factor Hull-White models, testing different machine learning architectures, investigating real-time calibration techniques, or extending the comparison to other financial instruments or markets.
\end{itemize}