Following the methodology outlined in the previous chapter, this section presents the empirical findings of the comparative analysis. The results are structured around the core evaluation criteria: calibration accuracy, computational speed, and the robustness of the derived Hull-White parameters. Through a series of tables, figures, and quantitative metrics, this chapter provides an objective assessment of how well each method replicates observed market prices and how they perform under different conditions. The aim is to deliver a clear, evidence-based comparison that highlights the respective strengths and weaknesses of the traditional and machine learning-based approaches.

\subsection{Interpretation of Principal Component Analysis on the Yield Curve}
The application of Principal Component Analysis (PCA) to the yield curve provides a powerful framework for decomposing the complex dynamics of interest rate movements into a small number of uncorrelated factors. The empirical results of the analysis indicate that the first three principal components collectively explain 99.84\% of the total variance observed in daily yield curve changes. This finding confirms that the term structure of interest rates can be effectively described by a limited set of underlying factors, each representing a distinct pattern of yield curve movement.

\begin{table}[H]
	\centering
	\label{tab:pca_variance}
	\begin{tabular}{lcc}
		\toprule
		Principal Component & Explained Variance (\%) & Cumulative Variance (\%) \\
		\midrule
		PC1 (Level)                  & 91.08                            & 91.08                             \\
		PC2 (Slope)                  & 8.16                             & 99.24                             \\
		PC3 (Curvature)              & 0.59                             & 99.84                             \\
		\bottomrule
	\end{tabular}
	\caption{Explained Variance of the Principal Components}
\end{table}

The first principal component (PC1) accounts for 91.08\% of the total variance and clearly emerges as the dominant driver of yield curve fluctuations. The loading plot of PC1 reveals positive loadings across all maturities, indicating that variations in this factor correspond to parallel shifts in the yield curve. In other words, changes in PC1 lead to simultaneous increases or decreases in yields across all tenors, representing the so-called level factor. This component captures broad-based movements in the general level of interest rates and is thus associated with macroeconomic influences such as monetary policy shifts and long-term inflation expectations.

The second principal component (PC2), explaining 8.16\% of the total variance, can be interpreted as the slope factor. Its loading structure is characterized by negative values for short maturities and positive values for longer maturities, reflecting an inverse relationship between short- and long-term rates. When this factor increases, the yield curve steepens—short-term yields decline while long-term yields rise. Conversely, a decrease in this component leads to a flattening of the curve. The slope factor therefore captures non-parallel shifts in the term structure and is closely related to expectations regarding future economic growth and central bank policy adjustments.

The third principal component (PC3) explains 0.59\% of the total variance and represents the curvature factor. Its loading pattern exhibits a distinct hump shape, with positive loadings at the short and long ends of the maturity spectrum and negative loadings in the intermediate maturities. This configuration indicates that variations in PC3 alter the concavity of the yield curve, producing so-called butterfly movements. Such changes often occur when medium-term interest rates move differently from short- and long-term rates, providing information about market expectations of medium-term monetary policy and term premia.

The empirical findings are consistent with the theoretical and empirical literature on interest rate modeling, particularly the work of \parencite[pp.~98--107]{Rebonato_2018}. The identification of level, slope, and curvature as the three principal components of the yield curve is a well-established result in fixed-income research. Their hierarchical importance—where the level factor dominates, followed by the slope and curvature factors—reflects a universal characteristic of yield curve dynamics across different markets and time periods.

The concentration of explanatory power within the first three components demonstrates the suitability of PCA for dimensionality reduction in yield curve modeling. Instead of modeling the dynamics of nine highly correlated interest rates, the analysis can be simplified to three orthogonal factors that capture nearly all relevant variation. This dimensionality reduction not only mitigates model complexity and overfitting risk but also enhances interpretability by linking statistical factors to economically meaningful concepts. Furthermore, such a factor-based representation supports practical applications in risk management, hedging, and scenario analysis, as the identified components correspond directly to the main sources of interest rate risk in the term structure.

\begin{itemize}
	\item \textbf{Calibration Performance Metrics:} Define and present the metrics used to quantify the "goodness of fit" for both traditional and ML-based calibration methods.
	      \begin{itemize}
		      \item Common metrics include Mean Squared Error (MSE), Root Mean Squared Error (RMSE), or Mean Absolute Error (MAE) between model-implied and market-observed swaption prices or implied volatilities.
		      \item Discuss any percentage errors or relative errors used.
	      \end{itemize}
	\item \textbf{Comparison of Calibration Speed:} Quantitatively compare the computational efficiency of each method.
	      \begin{itemize}
		      \item Provide average calibration times for the traditional algorithms (e.g., Levenberg-Marquardt) per market snapshot.
		      \item Detail the training time required for the neural network model.
		      \item Report the inference or prediction time for the neural network once trained, when applied to new market data.
		      \item Discuss the trade-offs, such as the initial higher computational cost of training the NN versus its potentially faster inference speed for subsequent calibrations.
	      \end{itemize}
	\item \textbf{Comparison of Accuracy and Robustness:} Analyze how accurately each method calibrated the Hull-White parameters and replicated observed market prices/volatilities.
	      \begin{itemize}
		      \item Present tables and graphs comparing model-implied swaption volatilities/prices against market values.
		      \item Discuss how well each method captures the various features of the volatility surface (e.g., skew, smile, term structure).
		      \item Evaluate the robustness of each method to noisy or incomplete market data, as well as its stability under different market regimes (e.g., high vs. low volatility periods).
	      \end{itemize}
	\item \textbf{Analysis of Hull-White Parameters:} Examine the derived mean reversion `a(t)` and volatility `$\sigma(t)$` parameters from both approaches.
	      \begin{itemize}
		      \item Use visualizations (e.g., time series plots or surface plots) to show how these parameters vary across time, option maturities, or swap tenors for each method.
		      \item Offer an interpretation of these differences, linking them to the underlying assumptions and characteristics of each calibration technique and their financial implications.
	      \end{itemize}
	\item \textbf{Sensitivity Analysis (Optional):} Investigate how sensitive each model's calibration results are to changes in input data, initial parameter guesses, or hyperparameter choices.
\end{itemize}
