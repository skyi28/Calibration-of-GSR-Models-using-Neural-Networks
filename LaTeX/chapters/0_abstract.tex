This thesis presents a novel, end-to-end machine learning framework for the calibration of the one-factor \ac{hw} model and provides a comprehensive comparison against the traditional \ac{lm} algorithm. The core innovation involves integrating a neural network directly with the QuantLib pricing engine. In a significant departure from antecedent methods that train networks to mimic pre-calibrated parameters, this framework forces the network to learn by minimizing the true pricing error, dynamically computing model-implied swaption volatilities and comparing them against market observations. To inform its predictions, the network leverages a rich feature set, including external market indicators and the principal components of the yield curve.

This direct optimization framework is calibrated and tested on a dataset of European \ac{atm} swaptions from the \ac{eur} market, covering the period from June 1, 2025, to August 31, 2025. The empirical results demonstrate that this approach yields a calibration tool that is superior in speed, accuracy, and robustness. Once trained, the network performs calibration in milliseconds, a computational speed-up of over 22,000 times compared to the \ac{lm} optimizer, effectively shifting the computational burden to a one-time offline phase. Furthermore, it produces economically sensible and stable parameters that react logically to simulated market shocks, whereas the traditional optimizer frequently suffers from numerical instability, with key parameters collapsing under stress scenarios.

The primary bottleneck of this advanced methodology is the substantial computational effort required for training, as the integration with the non-differentiable QuantLib engine necessitates numerical gradient approximations. This hybrid approach, while powerful, represents the main trade-off for achieving a more robust and accurate calibration tool. Despite this training challenge, the findings confirm that learning to minimize pricing error directly offers significant advantages, marking a promising advancement in financial model calibration.

\textit{JEL Classification: G13, C45}