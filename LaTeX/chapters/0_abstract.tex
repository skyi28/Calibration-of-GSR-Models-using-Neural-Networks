This thesis conducts a comprehensive comparison between the traditional Levenberg-Marquardt algorithm and a novel, end-to-end machine learning approach for the calibration of the one-factor Hull-White model. The core of the methodology involves a neural network trained to directly predict the parameters of the Hull-White model. To inform its predictions, the network leverages a rich set of inputs, including external market indicators and engineered features that capture the slope and curvature of the yield curve. In a significant departure from previous machine learning approaches, the network is integrated with the QuantLib pricing engine; its output parameters are dynamically plugged into QuantLib's pricing formula to compute model-implied at-the-money swaption volatilities. The learning process is driven by the direct minimization of the pricing error between these model-implied volatilities and observed market volatilities.

This novel, end-to-end training framework proves to be superior to antecedent methods where a neural network is trained on a predetermined target set of parameters previously calculated by an optimizer like the Levenberg-Marquardt algorithm. By learning the parameter mapping that is optimal for pricing rather than merely mimicking a traditional optimizer, the network demonstrates improved out-of-sample pricing accuracy and parameter stability.

The models are calibrated and tested on a comprehensive dataset of European at-the-money (ATM) swaptions from the Euro (EUR) market, covering the period from June 1, 2025, to August 31, 2025. Empirical results from this dataset reveal that the neural network consistently outperforms the traditional method in computational efficiency and parameter robustness. Once trained, the neural network performs calibration in milliseconds, achieving a computational speed-up of over 22,000 times compared to the Levenberg-Marquardt optimizer. This effectively shifts the main computational burden from a slow, real-time application to a one-time, offline training phase. Furthermore, the network produces economically sensible and stable parameters that react logically to simulated market shocks, whereas the traditional optimizer frequently suffers from numerical instability, with key parameters collapsing or exploding under stress scenarios.

The primary bottleneck of this advanced methodology is the substantial computational effort required for training. The integration of the differentiable TensorFlow framework with the non-differentiable QuantLib pricing engine necessitates the use of numerical gradient approximations. This hybrid approach, while powerful, is computationally intensive, representing the main trade-off for achieving a more robust and accurate calibration tool. Despite this training challenge, the findings confirm that the direct machine learning approach offers significant advantages in speed, accuracy, and robustness, marking a promising advancement in financial model calibration.

\textit{JEL Classification: G13, C45}