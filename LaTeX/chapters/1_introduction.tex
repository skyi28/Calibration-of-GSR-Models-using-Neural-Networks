Interest rate models are a cornerstone of modern quantitative finance, providing the essential framework for pricing derivatives, managing risk, and formulating hedging strategies \parencite{moysiadis2019calibrating}. Among the various term structure models, the one-factor \ac{hw} model has established itself as a benchmark due to its analytical tractability and its ability to fit the initial term structure of interest rates. Its application is widespread, from the valuation of simple options like caps and floors to more complex, path-dependent instruments such as Bermudan swaptions \parencite[p.~72]{brigo2006interest}. However, the utility of any model is fundamentally dependent on its calibration, the process of aligning its parameters with observed market prices to ensure that it reflects current market conditions and expectations.

The calibration of the \ac{hw} model, particularly its mean reversion and volatility parameters, presents a significant challenge. Traditional techniques are well established, relying on powerful numerical optimization algorithms like the \ac{lm} method to minimize the discrepancy between model and market prices. While this approach is a cornerstone of quantitative practice, its inherent properties create challenges for modern, high-frequency applications. As iterative numerical methods, they are by design computationally demanding, requiring significant time to converge. Furthermore, the non-convex nature of the calibration problem makes them susceptible to converging to local minima, potentially yielding suboptimal parameters that do not fully capture the market landscape, particularly in volatile regimes \parencite{vollrath2009calibration}. In an increasingly fast-paced environment that demands real-time pricing and risk management, this computational latency and potential for parameter instability create a critical bottleneck \parencite{lui2019nnfinancialmodelcalibration}.

The recent advancements in machine learning, and specifically in the field of \ac{nn}, offer a promising alternative to address these challenges. \ac{nn}s are exceptionally adept at learning complex, non-linear relationships directly from data. By training a \ac{nn} on historical market data and the corresponding "true" model parameters, it is possible to create a surrogate model that can approximate the calibration function almost instantaneously. This data-driven approach has the potential to transform the calibration process from a time-consuming optimization task into a rapid inference problem, thereby offering significant gains in speed and consistency \parencite{hernandez2016model}. The motivation for this research stems from the pressing need for more efficient and robust calibration techniques in an increasingly fast-paced financial world.

The primary aim of this thesis is to provide a comprehensive comparison between the traditional, optimization-based calibration of the one-factor \ac{hw} model and a novel, machine learning-based approach. To achieve this, the research will be guided by the following key questions:
\begin{enumerate}
	\item How does the accuracy of a \ac{nn}-based calibration, measured by the model's ability to replicate market swaption prices, compare to that of the traditional \ac{lm} algorithm?
	\item What is the quantitative difference in computational speed between the two methods, considering both the initial setup (training) and the ongoing application (inference)?
	\item How stable are the estimated model parameters over time?
\end{enumerate}

To ensure a focused and rigorous analysis, this study is subject to several delimitations. The scope is confined to the one-factor \ac{hw} model, calibrated using a comprehensive dataset of European \ac{atm} swaptions from the \ac{eur} market. The analysis covers a continuous time period from 01.06.2025 to 31.08.2025, providing a view of the model's performance in a recent market environment. The study does not extend to multi-factor models or other classes of interest rate derivatives.

This thesis is structured as follows. Section \ref{background} provides a detailed background on the theoretical foundations, including swaptions, yield curve bootstrapping, a thorough review of the \ac{hw} model and its traditional calibration, and an introduction to the core principles of the machine learning approach, outlining the foundational concepts of \ac{nn}s, their architecture, and learning mechanisms like the back-propagation algorithm. Section \ref{methodology} delineates the research methodology, covering the dataset and a multi-stage data preprocessing pipeline that includes yield curve bootstrapping , feature engineering via \ac{pca} , and feature scaling. It then details the implementation of the end-to-end machine learning framework , including the hyperparameter optimization conducted, the use of a custom asymmetric loss function, and the technical setup for integrating \texttt{TensorFlow} with \texttt{QuantLib}. Finally, the section establishes the intra-day hold-out set protocol and the error metrics, used for the comparative evaluation. Section \ref{results} presents the empirical results of the comparative analysis, focusing on pricing abilities, speed, and robustness. Section \ref{limitations} discusses the limitations of this study arising from the model choice, dataset, methodology, evaluation framework, and practical implementation. Finally, section \ref{conclusion} concludes the thesis by summarizing the key findings, discussing their implications for financial practitioners, and suggesting avenues for future research.