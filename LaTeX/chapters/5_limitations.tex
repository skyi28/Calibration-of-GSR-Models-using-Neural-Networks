This study is subject to several limitations arising from the model choice, dataset, methodology, evaluation framework, and practical implementation. The research exclusively focuses on the one-factor Hull-White model. While this model serves as a standard benchmark, it has inherent limitations in capturing complex yield curve dynamics, such as twists and butterfly movements, and in fitting the complete swaption volatility surface. Consequently, the findings may not directly generalize to more sophisticated multi-factor interest rate models. Additionally, the model is calibrated solely to European at-the-money swaptions, leaving its ability to price out-of-the-money or in-the-money options and to reproduce the volatility smile or skew unevaluated.  

The dataset employed introduces several constraints. The analysis covers only a short, recent period from June 1, 2025, to August 31, 2025, characterized by generally decreasing volatility and moderately rising rates. The performance across other market regimes, including periods of high volatility, financial crises, or near-zero interest rates, remains untested. Data were manually extracted from screenshots due to license restrictions, introducing potential transcription errors that would not exist with a direct API feed. Moreover, the use of mid-market quotes for swap rates and swaption volatilities ignores bid-ask spreads, which reflect transaction costs and liquidity, potentially affecting practical profitability and risk assessment. The model is trained on daily snapshots, which limits its ability to capture intraday movements, restricting applicability for real-time pricing or high-frequency trading. The research is confined to the Euro market, so the conclusions may not extend to other currency markets with different structures, liquidity profiles, and central bank policies.  

Methodologically, the end-to-end training approach that directly minimizes pricing errors using QuantLib is computationally intensive due to numerical gradient approximation, which may have limited hyperparameter tuning and model complexity. Static hyperparameters were used, although optimal settings may vary across market regimes, and no feature selection algorithm was employed, which could have further improved performance. Principal Component Analysis (PCA) was used for feature engineering, but as a linear dimensionality reduction technique, it may not capture non-linear relationships fully. Neural networks, despite SHAP-based interpretability, remain complex and less transparent than traditional optimizers, complicating economic interpretation. The study also does not evaluate hedge ratio stability or risk management performance; pricing accuracy alone may not indicate reliable hedge parameters. The focus on RMSE as a single error metric does not capture tail risks or the distribution of errors, and the artificial hold-out set used for the Levenberg-Marquardt method may slightly handicap its in-sample performance relative to real-world usage.  

From a practical perspective, the neural network approach requires substantial upfront investment in data collection, feature engineering, hyperparameter tuning, and initial training. While inference is fast, this setup cost may limit adoption. Long-term performance and the need for retraining could not be fully assessed, given the short test period. Model risk, governance, and regulatory considerations of deploying a neural network in a financial institution were not addressed. The study assumes liquid swaptions, leaving calibration performance in illiquid or sparse regions unexplored. These limitations collectively suggest that while the proposed approach demonstrates promising results, its applicability and robustness in broader, real-world settings require further investigation.
