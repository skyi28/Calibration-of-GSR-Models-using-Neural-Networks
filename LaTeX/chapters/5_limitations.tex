This study is subject to several limitations arising from the model choice, dataset, methodology, evaluation framework, and practical implementation. The research exclusively focuses on the one-factor \ac{hw} model. While this model serves as a standard benchmark, it has inherent limitations in capturing complex yield curve dynamics, such as twists and butterfly movements, and in fitting the complete swaption volatility surface. Consequently, the findings may not directly generalize to more sophisticated multi-factor interest rate models. Additionally, the model is calibrated solely to European \ac{atm}, leaving its ability to price \ac{otm} or \ac{itm} options and to reproduce the volatility smile or skew unevaluated.

The dataset employed introduces several constraints. The analysis covers only a short, recent period from June 1, 2025, to August 31, 2025, characterized by generally decreasing volatility and moderately rising rates. The performance across other market regimes, including periods of high volatility, financial crises, or near-zero interest rates, remains untested. Data were manually extracted from screenshots due to license restrictions, introducing potential transcription errors that would not exist with a direct \ac{api} feed. Moreover, the use of mid-market quotes for swap rates and swaption volatilities ignores bid-ask spreads, which reflect transaction costs and liquidity, potentially affecting practical profitability and risk assessment. The \ac{nn} is trained on daily snapshots, which limits its ability to capture intraday movements, restricting applicability for real-time pricing or high-frequency trading. The research is confined to the Euro market, so the conclusions may not extend to other currency markets with different structures, liquidity profiles, and central bank policies.

Methodologically, the evaluation framework itself contains significant constraints. **Perhaps the most critical limitation from a practical risk management perspective is the study's exclusive focus on pricing accuracy, while the stability and reliability of the resulting hedge parameters remain unevaluated.** A model can exhibit a low pricing error (RMSE) yet produce highly volatile hedge ratios if its calibrated parameters—and thus its derivatives with respect to market variables—fluctuate erratically over time. Such instability would render the model impractical for active hedging, as it would necessitate frequent and costly portfolio rebalancing that could erode any pricing advantages. In a production environment, predictable and stable hedge parameters are often more critical than marginal improvements in pricing accuracy. Therefore, while this thesis demonstrates the NN's superiority in replicating market prices, it does not confirm its utility as a robust tool for risk management, which remains an unaddressed question. Furthermore, the focus on \ac{rmse} as a single error metric does not capture tail risks or the distribution of errors, and the artificial hold-out set used for the \ac{lm} method may slightly handicap its in-sample performance relative to real-world usage.

From a technical and practical perspective, the end-to-end training approach is computationally intensive due to numerical gradient approximation, which may have limited hyperparameter tuning and model complexity. Static hyperparameters were used, although optimal settings may vary across market regimes, and no feature selection algorithm was employed. \ac{pca}, as a linear technique, may not capture non-linear relationships fully. Moreover, the \ac{nn} approach requires substantial upfront investment in data collection, feature engineering, and initial training. The long-term performance and need for retraining could not be fully assessed, and crucial aspects of model risk, governance, and regulatory considerations were not addressed. These limitations collectively suggest that while the proposed approach demonstrates promising results, its applicability and robustness in broader, real-world settings require further investigation.